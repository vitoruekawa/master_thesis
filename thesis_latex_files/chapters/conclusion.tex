\chapter{Conclusion and Future Work}
This thesis investigated the integration of renewable energy sources into power
systems, focusing on the challenge of decreased system inertia due to the
replacement of synchronous generators with inverter-based resources. The
adoption of grid-forming control strategies, particularly the Virtual
Synchronous Machine (VSM), was analyzed as a potential solution to emulate the
dynamics of traditional generators and maintain system stability.

Among various VSM topologies, the cascaded VSM was identified as the optimal
choice due to its grid-forming and supporting capabilities, and current
limiting features. The thesis compared the effectiveness of using different
synchronous generator models—2-axis, 1-axis, and classical models—to generate
reference signals for the VSM.

The simulation results revealed that the 2-axis model, despite offering an
additional parameter ($X_q'$), does not significantly impact system dynamics,
making its added complexity unnecessary. Conversely, the parameter $X_d'$ in the
1-axis model was found to have a substantial effect on system behavior. As a
result, the 1-axis model is recommended for VSM implementation when the goal is
to accurately replicate synchronous generator dynamics. The classical model, due
to its oversimplified nature, may not adequately capture the required dynamics.

In conclusion, this thesis supports the use of the 1-axis synchronous generator
model within the cascaded VSM framework for enhancing the dynamic response of
power systems with high renewable energy penetration. This approach simplifies
the VSM design while effectively emulating the inertia and damping
characteristics of conventional generators, contributing to the stability and
reliability of modern power systems. As an extension of this thesis, the
following topics are outlined:

\begin{itemize}
    \item \textbf{Passive and active damping:} investigating passive and active
    damping techniques within the VSM framework can further enhance system
    stability, particularly in damping out oscillations following disturbances.
    Passive damping involves inherent system design features, whereas active
    damping employs control strategies to dynamically adjust system parameters
    in response to oscillations. Future work could focus on optimizing these
    damping mechanisms to improve system robustness against a wide range of
    disturbances.
    \item \textbf{Load sharing capability:} a critical aspect of integrating
    multiple VSMs into the power grid is their ability to share load effectively
    and maintain system balance. Future studies could explore advanced control
    strategies that enable precise and equitable load sharing among VSMs, taking
    into account their dynamic capabilities and constraints. This research could
    also include the development of algorithms for real-time load distribution
    in response to changing grid conditions.
    \item \textbf{Detailed modeling of DC-source:} a critical aspect of
    integrating multiple VSMs into the power grid is their ability to share load
    effectively and maintain system balance. Future studies could explore
    advanced control strategies that enable precise and equitable load sharing
    among VSMs, taking into account their dynamic capabilities and constraints.
    This research could also include the development of algorithms for real-time
    load distribution in response to changing grid conditions.
    \item \textbf{Studies on other Grid Forming control methods:} while this
    thesis focused on the VSM approach, other grid-forming control methods offer
    potential benefits for integrating renewable energy sources into power
    systems. Future research could involve a comprehensive comparison of various
    grid-forming control strategies, evaluating their performance, scalability,
    and suitability for different grid configurations. This comparative analysis
    would help in identifying the most effective control solutions for ensuring
    grid stability and reliability in the face of increasing renewable
    penetration.
\end{itemize}

Each of these future research directions aims to build upon the findings of this
thesis, addressing key challenges and opportunities in the integration of
renewable energy sources and advanced control technologies into power systems.




