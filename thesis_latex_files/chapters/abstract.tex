\begin{eabstract}

In recent years, inverter-based resources (IBR) with renewable energy sources
(RES), such as solar and wind energy, have been developed as a solution for the
growing demand for energy and the challenges related to climate change. However,
as the penetration of IBRs increases and the participation of traditional large
and bulky synchronous generators in the energy generation decreases, many
research challenges arises. Among them, the loss of system inertia is one of the
most detrimental to power systems.

This thesis explores the grid-forming control strategy as a viable solution to
this challenge, with a focus on the Virtual Synchronous Machine (VSM) concept.
The VSM is designed to mimic the dynamic behavior of synchronous generators,
thereby contributing to system inertia and enhancing stability.

A comprehensive review of VSM topologies and control strategies is conducted,
and the cascaded virtual synchronous machine (CVSM) topology is identified for
its grid-forming and supporting capabilities, and inherent ability to limit the
converter's current. Then, the mathematical modeling of the voltage source
converter (VSC), together with its output filter and controllers enabling the 
VSM capabilities, is discussed. Different synchronous generator models - 2-axis,
1-axis, and classical - together with a virtual excitation system is proposed for 
generating the reference signals for the converter.

Simulation results in the WECC-9 bus system, reveal that while the 2-axis model
offers an additional degree of freedom through the parameter $X_q'$, its impact
on the system dynamics is minimal and can be overlooked. Conversely, the
parameter $X_d'$ in the 1-axis model is shown to have a significant effect on
system behavior, making the 1-axis model more appropriate for VSM applications
aimed at emulating synchronous generator dynamics. The findings suggest that the
2-axis model may introduce unnecessary complexity without tangible benefits, and
the classical model may oversimplify the dynamics.

In conclusion, this thesis advocates for the adoption of the 1-axis synchronous
generator model within the cascaded VSM framework to effectively replicate the
dynamics of traditional generators. This approach not only simplifies the VSM
design but also substantially contributes to the stability and reliability of
power systems undergoing rapid integration of renewable energy sources.

\end{eabstract}